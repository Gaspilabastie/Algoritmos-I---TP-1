\documentclass[10pt,a4paper]{article}

\usepackage[spanish,activeacute,es-tabla]{babel}
\usepackage[utf8]{inputenc}
\usepackage{ifthen}
\usepackage{listings}
\usepackage{dsfont}
\usepackage{subcaption}
\usepackage{amsmath}
\usepackage[strict]{changepage}
\usepackage[top=1cm,bottom=2cm,left=1cm,right=1cm]{geometry}%
\usepackage{color}%
\newcommand{\tocarEspacios}{%
	\addtolength{\leftskip}{3em}%
	\setlength{\parindent}{0em}%
}

% Especificacion de procs

\newcommand{\In}{\textsf{in }}
\newcommand{\Out}{\textsf{out }}
\newcommand{\Inout}{\textsf{inout }}

\newcommand{\encabezadoDeProc}[4]{%
	% Ponemos la palabrita problema en tt
	%  \noindent%
	{\normalfont\bfseries\ttfamily proc}%
	% Ponemos el nombre del problema
	\ %
	{\normalfont\ttfamily #2}%
	\
	% Ponemos los parametros
	(#3)%
	\ifthenelse{\equal{#4}{}}{}{%
		% Por ultimo, va el tipo del resultado
		\ : #4}
}

\newenvironment{proc}[4][res]{%
	
	% El parametro 1 (opcional) es el nombre del resultado
	% El parametro 2 es el nombre del problema
	% El parametro 3 son los parametros
	% El parametro 4 es el tipo del resultado
	% Preambulo del ambiente problema
	% Tenemos que definir los comandos requiere, asegura, modifica y aux
	\newcommand{\requiere}[2][]{%
		{\normalfont\bfseries\ttfamily requiere}%
		\ifthenelse{\equal{##1}{}}{}{\ {\normalfont\ttfamily ##1} :}\ %
		\{\ensuremath{##2}\}%
		{\normalfont\bfseries\,\par}%
	}
	\newcommand{\asegura}[2][]{%
		{\normalfont\bfseries\ttfamily asegura}%
		\ifthenelse{\equal{##1}{}}{}{\ {\normalfont\ttfamily ##1} :}\
		\{\ensuremath{##2}\}%
		{\normalfont\bfseries\,\par}%
	}
	\renewcommand{\aux}[4]{%
		{\normalfont\bfseries\ttfamily aux\ }%
		{\normalfont\ttfamily ##1}%
		\ifthenelse{\equal{##2}{}}{}{\ (##2)}\ : ##3\, = \ensuremath{##4}%
		{\normalfont\bfseries\,;\par}%
	}
	\renewcommand{\pred}[3]{%
		{\normalfont\bfseries\ttfamily pred }%
		{\normalfont\ttfamily ##1}%
		\ifthenelse{\equal{##2}{}}{}{\ (##2) }%
		\{%
		\begin{adjustwidth}{+5em}{}
			\ensuremath{##3}
		\end{adjustwidth}
		\}%
		{\normalfont\bfseries\,\par}%
	}
	
	\newcommand{\res}{#1}
	\vspace{1ex}
	\noindent
	\encabezadoDeProc{#1}{#2}{#3}{#4}
	% Abrimos la llave
	\par%
	\tocarEspacios
}
{
	% Cerramos la llave
	\vspace{1ex}
}

\newcommand{\aux}[4]{%
	{\normalfont\bfseries\ttfamily\noindent aux\ }%
	{\normalfont\ttfamily #1}%
	\ifthenelse{\equal{#2}{}}{}{\ (#2)}\ : #3\, = \ensuremath{#4}%
	{\normalfont\bfseries\,;\par}%
}

\newcommand{\pred}[3]{%
	{\normalfont\bfseries\ttfamily\noindent pred }%
	{\normalfont\ttfamily #1}%
	\ifthenelse{\equal{#2}{}}{}{\ (#2) }%
	\{%
	\begin{adjustwidth}{+2em}{}
		\ensuremath{#3}
	\end{adjustwidth}
	\}%
	{\normalfont\bfseries\,\par}%
}

% Tipos

\newcommand{\nat}{\ensuremath{\mathds{N}}}
\newcommand{\ent}{\ensuremath{\mathds{Z}}}
\newcommand{\real}{\ensuremath{\mathds{R}}}
\newcommand{\float}{\ensuremath{\mathds{R}}}
\newcommand{\bool}{\ensuremath{\mathsf{Bool}}}
\newcommand{\cha}{\ensuremath{\mathsf{Char}}}
\newcommand{\str}{\ensuremath{\mathsf{String}}}
\newcommand{\metavariable}{\ensuremath{\mathsf{v_{0}}}}
\newcommand{\precondicion}{\ensuremath{\mathsf{P_{c}}}}
\newcommand{\funcionDecresiente}{\ensuremath{\mathsf{f_{v}}}}
\newcommand{\postcondicion}{\ensuremath{\mathsf{Q_{c}}}}
\newcommand{\reemplazo}{\ensuremath{\mathsf{I_{i+1}^i}}}

\newcommand{\apuestac}{\ensuremath{\mathsf{apuesta_{c}}}}
\newcommand{\apuestas}{\ensuremath{\mathsf{apuesta_{s}}}}
\newcommand{\pagoc}{\ensuremath{\mathsf{pago_{c}}}}
\newcommand{\pagos}{\textit{\ensuremath{\mathsf{pago_{s}}}}}

% Logica

\newcommand{\True}{\ensuremath{\mathrm{true}}}
\newcommand{\False}{\ensuremath{\mathrm{false}}}
\newcommand{\Then}{\ensuremath{\rightarrow}}
\newcommand{\Iff}{\ensuremath{\leftrightarrow}}
\newcommand{\implica}{\ensuremath{\longrightarrow}}
\newcommand{\IfThenElse}[3]{\ensuremath{\mathsf{if}\ #1\ \mathsf{then}\ #2\ \mathsf{else}\ #3\ \mathsf{fi}}}
\newcommand{\y}{\land}
\newcommand{\yLuego}{\land _L}
\newcommand{\oLuego}{\lor _L}
\newcommand{\implicaLuego}{\implica _L}

\newcommand{\cuantificador}[5]{%
	\ensuremath{(#2 #3: #4)\ (%
		\ifthenelse{\equal{#1}{unalinea}}{
			#5
		}{
			$ % exiting math mode
			\begin{adjustwidth}{+2em}{}
				$#5$%
			\end{adjustwidth}%
			$ % entering math mode
		}
		)}
}

\newcommand{\existe}[4][]{%
	\cuantificador{#1}{\exists}{#2}{#3}{#4}
}
\newcommand{\paraTodo}[4][]{%
	\cuantificador{#1}{\forall}{#2}{#3}{#4}
}

%listas

\newcommand{\TLista}[1]{\ensuremath{seq \langle #1\rangle}}
\newcommand{\lvacia}{\ensuremath{[\ ]}}
\newcommand{\lv}{\ensuremath{[\ ]}}
\newcommand{\longitud}[1]{\ensuremath{|#1|}}
\newcommand{\cons}[1]{\ensuremath{\mathsf{addFirst}}(#1)}
\newcommand{\indice}[1]{\ensuremath{\mathsf{indice}}(#1)}
\newcommand{\conc}[1]{\ensuremath{\mathsf{concat}}(#1)}
\newcommand{\cab}[1]{\ensuremath{\mathsf{head}}(#1)}
\newcommand{\cola}[1]{\ensuremath{\mathsf{tail}}(#1)}
\newcommand{\sub}[1]{\ensuremath{\mathsf{subseq}}(#1)}
\newcommand{\en}[1]{\ensuremath{\mathsf{en}}(#1)}
\newcommand{\cuenta}[2]{\mathsf{cuenta}\ensuremath{(#1, #2)}}
\newcommand{\suma}[1]{\mathsf{suma}(#1)}
\newcommand{\twodots}{\ensuremath{\mathrm{..}}}
\newcommand{\masmas}{\ensuremath{++}}
\newcommand{\matriz}[1]{\TLista{\TLista{#1}}}
\newcommand{\seqchar}{\TLista{\cha}}

\renewcommand{\lstlistingname}{Código}
\lstset{% general command to set parameter(s)
	language=Java,
	morekeywords={endif, endwhile, skip},
	basewidth={0.47em,0.40em},
	columns=fixed, fontadjust, resetmargins, xrightmargin=5pt, xleftmargin=15pt,
	flexiblecolumns=false, tabsize=4, breaklines, breakatwhitespace=false, extendedchars=true,
	numbers=left, numberstyle=\tiny, stepnumber=1, numbersep=9pt,
	frame=l, framesep=3pt,
	captionpos=b,
}

\usepackage{caratula}
\usepackage[spanish]{babel} 
\usepackage{amsfonts} 
\usepackage[dvipsnames]{xcolor}

\titulo{Trabajo prático 1: Especificación y WP}
%\subtitulo{Subtítulo del tp}

\fecha{\today}

\materia{Algoritmos y Estructuras de Datos}
\grupo{Grupo JDDBVCNLUCVDVTZLNSIH}

\integrante{Labastié, Gaspar}{660/23}{gaspilabastie@gmail.com}
\integrante{Rugo, Julian}{1414/23}{julianrugo22@gmail.com}
\integrante{Torres, Emiliano}{80/23}{emilianomtorres1@gmail.com}
\integrante{Vanotti, Franco}{464/23}{fvanotti15@gmail.com}

\graphicspath{{../static/}}

\setcounter{section}{-1}

\begin{document}

\maketitle

\section{Aclaraciones generales} 


\begin{itemize}
	\item Los índices de las listas recursos, cooperan, trayectorias, apuestas, pagos, eventos representa el identificador de los
    individuos.
	\item recursos: seq⟨\real⟩ es la lista con el recurso de cada individuo.
	\item cooperan: seq⟨Bool⟩ es la lista que indica T rue si el individuo en dicha posición coopera.
	\item trayectorias: seq⟨seq⟨\real⟩⟩ indica para cada individuo, en cada paso de tiempos, cuántos recursos (\real) cuenta.
	\item eventos: seq⟨seq⟨\nat⟩⟩ indica para cada individuo, en cada paso temporal, qué evento le ha tocado.
	\item apuestas: seq⟨seq⟨\real⟩⟩ indica para cada individuo, para cada evento posible (numerados desde 0), cuánto apostará.
	\item pagos: seq⟨seq⟨\real⟩⟩ indica para cada individuo, para cada evento, cuánto se pagará. Notar que a diferencia del ejemplo, estams resolviendo un caso más general donde el pago de cada evento puede diferir por individuo.
	\item Las personas que no \textit{cooperan} no aportan nada al fondo monetario común.
	\item Los \textit{recursos} iniciales son positivos.
	\item Todos los \textit{pagos} son positivos.
	\item Las \textit{apuestas} de los individuos representan la proporción de los recursos que los individuos invierten a cada una de los eventos posibles. Notar nuevamente que a diferencia del ejemplo, en este caso más general, podríamos tener apuestas distintas para cada evento por cada individuo.
	\item Cada individuo apuesta siempre el mismo porcentaje por cada evento posible (es decir, el mismo número en cada paso temporal). Por ejemplo, si tenemos dos eventos; cara y ceca y apuesta 0, 4 por cara y 0,6 por seca, en cada paso temporal apostará esas proporciones.
    \item Se considera al número 0 como parte de \nat.
    \item Debido a la ambigüedad presente en el ejercico sobre especificación en cuanto a si un jugador puede o no puede apostar todo sus recursos a un unico evento, y por ende que la apuesta de dicho jugador a un evento sea igual a 0,  nosotros tomamos bajo nuestro criterio personal que eso si es posible, y se llevo a cabo el ejercicio considerando dicha posibilidad.
    
\end{itemize}


\section{Especificación} 

\begin{enumerate}
    \item \textbf{redistribucionDeLosFrutos}: Calcula los recursos que obtiene cada uno de los individuos luego de que se redistribuyen
    los recursos del fondo monetario común en partes iguales. El fondo monetario común se compone de la suma de \textit{recursos} iniciales aportados por todas las personas que \textit{cooperan}. La salida es la lista de recursos que tendrá cada jugador.

    \begin{proc}{redistribucionDeLosFrutos}{\In \textit{recursos} : seq⟨\real⟩, \In \textit{cooperan} : \TLista{Bool}}
    {seq⟨\real⟩}
        \requiere{|{\textit{recursos}}\vert > 0
        \y |{\textit{recursos}}\vert = |{\textit{cooperan}}\vert
        \y \\ 
        \paraTodo[unalinea]{i}{\ent}
        {0 \leq i < |{\textit{recursos}}\vert \implicaLuego \textit{recursos}[i] > 0
        }
        }


        \asegura{|{\textit{res}}\vert = |{\textit{recursos}}\vert 
        \yLuego 
         nuevosRecursosCooperan (\textit{recursos}, \textit{cooperan}, \textit{res}) 
         \y 
        \\ nuevosRecursosNoCooperan (\textit{recursos}, \textit{cooperan}, \textit{res})}

    \end{proc}
    

    \pred{nuevosRecursosCooperan}{\textit{recursos} : seq⟨\real⟩, \textit{cooperan} : \TLista{Bool}, \textit{res} : seq⟨\real⟩} 
    {\paraTodo[unalinea]{i}{\ent}
    {0 \leq i < |{\textit{recursos}}\vert \yLuego  \textit{cooperan}[i] = \True \implicaLuego res[i] = 
    distribuci\acute{o}nFondoCom\acute{u}n(\textit{recursos}, \textit{cooperan})}}
    
    \pred{nuevosRecursosNoCooperan}{\textit{recursos} : seq⟨\real⟩, \textit{cooperan} : \TLista{Bool}, \textit{res} : seq⟨\real⟩} 
    {\paraTodo[unalinea]{i}{\ent}
    {0 \leq i < |{\textit{recursos}}\vert \yLuego   \textit{cooperan}[i] = \False \implicaLuego res[i] = 
    distribuci\acute{o}nFondoCom\acute{u}n(\textit{recursos}, \textit{cooperan}) + \textit{recursos}[i]}}

    \aux{distribuciónFondoComún}{\textit{recursos} : 
    seq⟨\real⟩, \textit{cooperan} : \TLista{Bool}}{\real}
    {\\(\sum_{i = 0}^{|{\textit{recursos}}\vert-1})
    (if \hspace{2mm} \textit{cooperan}[i] = \True \hspace{2mm} then \hspace{2mm} \frac{\textit{recursos}[i]}{|{\textit{recursos}}\vert} \hspace{2mm} else \hspace{2mm} 0 \hspace{2mm} fi)}
    
    \clearpage

    \item \textbf{trayectoriaDeLosFrutosIndividualesALargoPlazo}: Actualiza (In/Out) la lista de \textit{trayectorias} de los los recursos de cada uno de los individuos. Inicialmente, cada una de las trayectorias (listas de recursos) contiene un único elemento que representa los recursos iniciales del individuo. El procedimiento agrega a las \textit{trayectorias} los recursos que los individuos van obteniendo a medida que se van produciendo los resultados de los \textit{eventos} en función de la lista de \textit{pagos} que le ofrece la naturaleza (o casa de apuestas) a cada uno de los individuos, las \textit{apuestas} (o inversiones) que realizan los individuos en cada paso temporal, y la lista de individuos que \textit{cooperan} aportando al fondo monetario común.
    
    \begin{proc}{trayectoriaDeLosFrutosIndividualesALargoPlazo}{\Inout \textit{trayectorias}:
        seq⟨seq⟨\real⟩⟩, \In \textit{cooperan} : \TLista{Bool}, \In \textit{apuestas}: seq⟨seq⟨\real⟩⟩, \In \textit{pagos}: seq⟨seq⟨\real⟩⟩, 
        \In \textit{eventos}: seq⟨seq⟨\nat⟩⟩}
    {}
        \requiere{
            |{\textit{trayectorias}}\vert = |{\textit{cooperan}}\vert = |{\textit{apuestas}}\vert = |{\textit{pagos}}\vert = |{\textit{eventos}}\vert \y \\
            \textit{trayectorias} = \textit{$Trayectorias_{0}$} \yLuego \\
            (\forall i : \ent) (0 \leq i <|{\textit{trayectorias}}\vert \implicaLuego (
                |{\textit{trayectorias}[i]}\vert =  1 \y \textit{trayectorias}[i][0] >  0
            )) \y |{\textit{pagos}}\vert > 0 \y \\
            (\forall k, l : \ent)(0 \leq k, l < |{\textit{apuestas}}\vert \implicaLuego |{\textit{apuestas}[k]}\vert = |{\textit{apuestas}[l]}\vert) \y \\
            (\forall i : \ent) ((0 \leq i < |{\textit{apuestas}}\vert \implicaLuego sumarApuestasIndividuo(\textit{apuestas}[i]) =1) \yLuego \\
            (\forall j : \ent) (0 \leq j < |{\textit{apuestas}[i]}\vert \implicaLuego 0 \leq \textit{apuestas}[i][j] \leq 1)) \y \\
            (\forall k, l : \ent)(0 \leq k, l < |{\textit{pagos}}\vert \implicaLuego |{\textit{pagos}[k]}\vert = |{\textit{pagos}[l]}\vert) \y \\
            (\forall i : \ent) (0 \leq i < |{\textit{pagos}}\vert \implicaLuego 
            (\forall j : \ent)(0 \leq j < |{\textit{pagos}[i]}\vert \implicaLuego 0 < \textit{pagos}[i][j]))
            \y \\
            (\forall i : \ent) (0 \leq i < |{\textit{pagos}}\vert \implicaLuego (\forall j : \ent)(0 \leq j < |{\textit{apuestas}}\vert \implicaLuego 
            |{\textit{pagos}[i]}\vert = |{\textit{apuestas}[j]}\vert))
            \y \\
            (\forall i : \ent)(0 \leq i < |{\textit{eventos}}\vert \implicaLuego |{\textit{eventos}[i]}\vert > 0)
            \y \\
            (\forall i : \ent)(0 \leq i < |{\textit{eventos}}\vert \implicaLuego |{\textit{eventos}[0]}\vert = |{\textit{eventos}[i]}\vert) \y \\
            (\forall i : \ent) (0 \leq i < |{\textit{eventos}}\vert \implicaLuego 
            (\forall j : \ent)(0 \leq j < |{\textit{eventos}[i]}\vert \implicaLuego 0 \leq \textit{eventos}[i][j] < |{\textit{pagos}[i]}\vert))
        }

        \asegura{
            |{\textit{trayectorias}}\vert = |{\textit{eventos}}\vert \yLuego
            (\forall i : \ent)(0 \leq i < |{\textit{trayectorias}}\vert \implicaLuego \textit{trayectorias}[i][0] = \textit{$Trayectorias_{0}$}[i][0]) \y \\
            (\forall i : \ent)(0 \leq i < |{\textit{trayectorias}}\vert \implicaLuego |{\textit{trayectorias}[i]}\vert = |{\textit{eventos}[i]}\vert +1) \y \\
            (\forall i : \ent) (0 \leq i < |{\textit{trayectorias}}\vert \implicaLuego 
            (\forall j : \ent)(0 < j < |{\textit{trayectorias}[i]}\vert \implicaLuego \\(trayectoriasCooperan(\textit{trayectorias}, \textit{cooperan}, \textit{apuestas}, \textit{pagos}, \textit{eventos}) \y \\
            trayectoriasNoCooperan(\textit{trayectorias}, \textit{cooperan}, \textit{apuestas}, \textit{pagos}, \textit{eventos}))))
            }
    \end{proc}

    \pred{trayectoriasCooperan}{\textit{trayectorias} : seq⟨seq⟨\real⟩⟩, \textit{cooperan} : seq⟨Bool⟩, \textit{apuestas} : seq⟨seq⟨\real⟩⟩, \textit{pagos} : seq⟨seq⟨\real⟩⟩, \textit{eventos} : seq⟨seq⟨\ent⟩⟩}
    {\textit{cooperan}[i] = \True \implicaLuego \textit{trayectorias}[i][j] = distribuci\acute{o}nFondoCom\acute{u}nTrayectoria(\textit{trayectorias}, \textit{cooperan}, \textit{apuestas}, \\\textit{pagos}, \textit{eventos}, j)}

    \pred{trayectoriasNoCooperan}{\textit{trayectorias} : seq⟨seq⟨\real⟩⟩, \textit{cooperan} : seq⟨Bool⟩, \textit{apuestas} : seq⟨seq⟨\real⟩⟩, \textit{pagos} : seq⟨seq⟨\real⟩⟩, \textit{eventos} : seq⟨seq⟨\ent⟩⟩} 
    {\textit{cooperan}[i] = \False \implicaLuego \textit{trayectorias}[i][j] = distribuci\acute{o}nFondoCom\acute{u}nTrayectoria(\textit{trayectorias}, \textit{cooperan}, \textit{apuestas},\\ \textit{pagos}, \textit{eventos}, j) + trayectoria[i][j-1]*gananciaIndividuo(apuestas[i], pagos[i],\\ eventos[i][j-1])}

    \aux{sumarApuestasIndividuo}{\textit{apuestasIndividuo} : seq⟨\real⟩}{\real}
    {\\(\sum_{n = 0}^{|{\textit{apuestasIndividuo}}\vert-1}) (\textit{apuestasIndividuo}[n])}

    \aux {gananciaIndividuo}{\textit{apuestaIndividuo}: seq⟨\real⟩, \textit{pagosIndividuo}: seq⟨\real⟩, \textit{resultadoEventoIndividuo}: \real}{\real}
    {\\\textit{apuestaIndividuo}[\textit{resultadoEventoIndividuo}] * \textit{pagosIndividuo}[\textit{resultadoEventoIndividuo}]}

    \aux{recolecciónFondoComún}{\textit{trayectorias}: seq⟨seq⟨\real⟩⟩, \textit{cooperan} : \TLista{Bool}, \textit{apuestas}: seq⟨seq⟨\real⟩⟩, \textit{pagos}: seq⟨seq⟨\real⟩⟩, 
    \textit{eventos}: seq⟨seq⟨\nat⟩⟩, j : \ent}{\real}{\\
        (\sum_{i = 0}^{j-1}){\\
        (if \hspace{1mm}  \textit{cooperan}[i] = \True \hspace{1mm} then \hspace{1mm} \textit{trayectorias}[i][j-1]*gananciaIndividuo(\textit{apuestas}[i], \textit{pagos}[i], \textit{eventos}[i][j-1]) \hspace{1mm} else  \hspace{1mm} 0 \hspace{1mm} fi)}
    }

    \aux{distribuciónFondoComúnTrayectoria}{\textit{trayectorias}: seq⟨seq⟨\real⟩⟩, \textit{cooperan} : \TLista{Bool}, \textit{apuestas}: seq⟨seq⟨\real⟩⟩, \textit{pagos}: seq⟨seq⟨\real⟩⟩, 
    \textit{eventos}: seq⟨seq⟨\nat⟩⟩, j : \ent}{\real}{\\
        \frac{(recolecci\acute{o}nFondoCom\acute{u}n(\textit{trayectorias}, \textit{cooperan}, \textit{apuestas}, \textit{pagos}, \textit{eventos}, j))}{|{\textit{trayectorias}}\vert}
    }

    \clearpage 

    \item \textbf{trayectoriaExtrañaEscalera} Esta función devuelve \textit{True} sii en la trayectoria de un individuo existe un único punto mayor a sus vecinos (llamado máximo local). Un elemento es máximo local si es mayor estricto que sus vecinos inmediatos.
    
    \begin{proc}{trayectoriaExtrañaEscalera}{\In \textit{trayectoria}: seq⟨\real⟩} {Bool}
        \requiere{
            |{\textit{trayectoria}}\vert > 0 \y 
            {\paraTodo[unalinea]{i}{\ent}
            {0 \leq i < |{\textit{trayectoria}}\vert \implicaLuego trayectoria[i] \geq 0}}
        }

        \asegura{
            res = \True \longleftrightarrow \\ 
            (\exists m : \ent) ((\forall i : \ent)
            ((0 \leq i, m < |{\textit{trayectoria}}\vert \y i \neq m)
            \implicaLuego \textit{trayectoria}[m] > \textit{trayectoria}[i])
            \yLuego \\
            (\forall n : \ent)(
                0 \leq n \leq m \implicaLuego \textit{trayectoria}[n] > \textit{trayectoria}[n - 1])
            \y \\
            (\forall n : \ent)(
                m \leq n < |{\textit{trayectoria}}\vert-1 \implicaLuego \textit{trayectoria}[n] > \textit{trayectoria}[n + 1])
            )
            } 

    \end{proc}

    \item \textbf{individuoDecideSiCooperarONo} Un \textit{individuo} actualiza su comportamiento cooperativo / no-cooperativo \\(\textit{cooperan}[\textit{individuo}]) en función de los recursos iniciales, de quienes \textit{cooperan}, de los \textit{pagos} que se le ofrecen a cada individuo, de las inversiones o \textit{apuestas} de cada individuo, y del resultado de los \textit{eventos} que recibe cada individuo, eligiendo el comportamiento que maximiza sus recursos individuales luego de que ocurren todos los eventos.

    \begin{proc}{individuoDecideSiCooperarONo}{\In \textit{individuo}:\nat, \In \textit{recursos}: seq⟨\real⟩, \Inout \textit{cooperan}: seq⟨Bool⟩, \In \textit{apuestas}: seq⟨seq⟨\real⟩⟩,
        \In \textit{pagos}: seq⟨seq⟨\real⟩⟩, \In \textit{eventos}: seq⟨seq⟨\nat⟩⟩} {}
        \requiere{
            |{\textit{recursos}}\vert > 0 \y |{\textit{recursos}}\vert = |{\textit{cooperan}}\vert = |{\textit{apuestas}}\vert = |{\textit{pagos}}\vert = |{\textit{eventos}}\vert \y \\
            \textit{cooperan} =  \textit{$Cooperan_{0}$} \y \\
            0 \leq \textit{individuo} < |{\textit{recursos}}\vert \yLuego \\
            (\forall i : \ent)(0 \leq i < |{\textit{recursos}}\vert \implicaLuego \textit{recursos}[i] > 0) \y \\
            (\forall i : \ent)(0 \leq i < |{\textit{eventos}}\vert \implicaLuego |{\textit{eventos}[i]}\vert > 0) \y \\
            (\forall i : \ent)(0 \leq i < |{\textit{eventos}}\vert \implicaLuego |{\textit{eventos}[i]}\vert = |{\textit{eventos}[0]}\vert) \y \\
            (\forall i : \ent) (0 \leq i < |{\textit{eventos}}\vert \implicaLuego 
            (\forall j : \ent)(0 \leq j < |{\textit{eventos}[i]}\vert \implicaLuego 0 \leq \textit{eventos}[i][j] < |{\textit{pagos}[i]}\vert)) \y \\
            (\forall k, l : \ent)(0 \leq k, l < |{\textit{apuestas}}\vert \implicaLuego |{\textit{apuestas}[k]}\vert = |{\textit{apuestas}[l]}\vert) \y \\
            (\forall i : \ent) ((0 \leq i < |{\textit{apuestas}}\vert \implicaLuego sumarApuestasIndividuo(\textit{apuestas}[i]) =1) \yLuego \\
            (\forall j : \ent) (0 \leq j < |{\textit{apuestas}[i]}\vert \implicaLuego 0 \leq \textit{apuestas}[i][j] \leq 1)) \y \\
            (\forall k, l : \ent)(0 \leq k, l < |{\textit{pagos}}\vert \implicaLuego |{\textit{pagos}[k]}\vert = |{\textit{pagos}[l]}\vert) \y \\
            (\forall i : \ent) (0 \leq i < |{\textit{pagos}}\vert \implicaLuego 
            (\forall j : \ent)(0 \leq j < |{\textit{pagos}[i]}\vert \implicaLuego 0 < \textit{pagos}[i][j])) \y \\
            (\forall i : \ent) (0 \leq i < |{\textit{pagos}}\vert \implicaLuego (\forall j : \ent)(0 \leq j < |{\textit{apuestas}}\vert \implicaLuego 
            |{\textit{pagos}[i]}\vert = |{\textit{apuestas}[j]}\vert))
        }
        \asegura{
            |\textit{cooperan}| =  |\textit{$Cooperan_{0}$}| \yLuego \\ %probar si || es igual a | \vert
            (\forall i : \ent)(0 \leq i < |\textit{$Cooperan_{0}$}| \y i \neq \textit{individuo} \implicaLuego \textit{cooperan}[i] = \textit{$Cooperan_{0}$}[i]) \y \\
            (\exists \textit{cooperanAlt}: seq⟨Bool⟩)((|{\textit{cooperanAlt}}\vert = |\textit{$Cooperan_{0}$}\vert \yLuego \\
            (\forall i : \ent)(0 \leq i < |{\textit{cooperanAlt}}\vert \y i \neq \textit{individuo} \implicaLuego \textit{cooperanAlt}[i] = \textit{$Cooperan_{0}$}[i]) \y \\ (\textit{cooperanAlt}[\textit{individuo}] = \neg\textit{$Cooperan_{0}$}[\textit{individuo}])) \y \\
            (\exists \textit{trayectoria} : seq\langle seq\langle \real\rangle\rangle)(|{\textit{trayectoria}}\vert = |{\textit{eventos}}\vert \yLuego primeroTieneARecursos(\textit{recursos}, \textit{trayectoria}) \y \\ moduloEventosMasUno(\textit{eventos}, \textit{trayectoria}) \y elementosDeT(\textit{apuestas},
            \textit{pagos}, \textit{eventos}, \textit{$Cooperan_{0}$}, \textit{trayectoria}) \y
            (\exists \textit{trayectoriaAlt} : seq\langle seq\langle \real\rangle\rangle)(|{\textit{trayectoriaAlt}}\vert = |{\textit{eventos}}\vert \yLuego primeroTieneARecursos(\textit{recursos}, \textit{trayectoriaAlt}) \y moduloEventosMasUno(\textit{eventos}, \textit{trayectoriaAlt}) \y \\
            elementosDeT(\textit{apuestas}, \textit{pagos}, \textit{eventos}, \textit{cooperanAlt}, \textit{trayectoriaAlt}) \yLuego \\
            \textit{cooperan}[\textit{individuo}] = mejorEleccion(\textit{trayectoria}, \textit{trayectoriaAlt},\textit{Individuo}))
            ))
        }

    \end{proc}

    \pred{primeroTieneARecursos}{\textit{recursos}:seq⟨\real⟩, \textit{trayectoria}:seq⟨seq⟨\real⟩⟩}
    {(\forall i:\ent)(0 \leq i <|{\textit{trayectoria}}\vert \implicaLuego \textit{trayectoria}[i][0] = \textit{recursos}[i])}

    \pred{moduloEventosMasUno}{\textit{eventos}:seq⟨seq⟨\real⟩⟩, \textit{trayectoria}:seq⟨seq⟨\real⟩⟩}
    {(\forall i :\ent)(0 \leq i <|{\textit{trayectoria}}\vert \implicaLuego |{\textit{trayectoria}[i]}\vert = |{\textit{eventos}}\vert +1)}

    \pred{elementosDeT}{\textit{apuestas}:seq⟨seq⟨\real⟩⟩,\textit{pagos}:seq⟨seq⟨\real⟩⟩, \textit{eventos}:seq⟨seq⟨\real⟩⟩, \\
    \textit{cooperan}:seq⟨Bool⟩, \textit{trayectoria}:seq⟨seq⟨\real⟩⟩}{(\forall i: \ent)(0 \leq i <|{\textit{trayectoria}}\vert \implicaLuego (\forall j: \ent)(0 \leq j <||{\textit{trayectoria}[i]}\vert \implicaLuego \\
    trayectoriasCooperan(\textit{trayectorias}, \textit{cooperan}, \textit{apuestas}, \textit{pagos}, \textit{eventos}) \y \\
    trayectoriasNoCooperan(\textit{trayectorias}, \textit{cooperan}, \textit{apuestas}, \textit{pagos}, \textit{eventos})))}

    \aux{mejorEleccion}{\textit{trayectoria}:seq⟨seq⟨\real⟩⟩,\textit{trayectoriaAlt}:seq⟨seq⟨\real⟩⟩,\textit{Individuo}}{Bool}
    {\\(if \hspace{2mm} \textit{trayectoria}[\textit{individuo}][|{\textit{trayectoria}[\textit{Individuo}]}\vert -1] \geq \textit{trayectoriaAlt}[\textit{individuo}][|{\textit{trayectoriaAlt}[\textit{Individuo}]}\vert -1] \hspace{2mm} then \\
    \textit{$Cooperan_{0}$}[\textit{individuo}] \hspace{2mm} else \hspace{2mm} \textit{CooperanAlt}[\textit{individuo}] \hspace{2mm} fi)}
    

     \clearpage

     \item \textbf{individuoActualizaApuesta} Un \textit{individuo} actualiza su apuesta (\textit{apuestas}[\textit{individuo}]) en función de los \textit{recursos} iniciales, de la lista de individuos que \textit{cooperan} , de los \textit{pagos} que se le ofrecen a cada individuo, de las inversiones o \textit{apuestas} de cada individuo y del resultado de los eventos que recibe cada individuo, eligiendo la apuesta que maximiza sus recursos individuales luego de que ocurren todos los eventos.

     \begin{proc}{individuoActualizaApuesta}{\In \textit{individuo}:\nat, \In \textit{recursos}: seq⟨\real⟩, \In \textit{cooperan}: seq⟨Bool⟩, \Inout \textit{apuestas}: seq⟨seq⟨\real⟩⟩,
        \In \textit{pagos}: seq⟨seq⟨\real⟩⟩, \In \textit{eventos}: seq⟨seq⟨\nat⟩⟩}{}
        \requiere{|{\textit{recursos}}\vert > 0 \y |{\textit{recursos}}\vert = |{\textit{cooperan}}\vert = |{\textit{apuestas}}\vert = |{\textit{pagos}}\vert = |{\textit{eventos}}\vert \yLuego \\
        (\forall i: \ent)(0 \leq i<|{\textit{recursos}}\vert \implicaLuego \textit{recursos[i]} > 0) \y \\
        \textit{apuestas}=\textit{$Apuestas_{0}$} \y \\
        (\forall i,j: \ent)(0 \leq i,j<|{\textit{apuestas}}\vert \implicaLuego |{\textit{apuestas}[i]}\vert = |{\textit{apuestas}[j]}\vert) \y \\
        (\forall i :  \ent)(0 \leq i <|{\textit{apuestas}}\vert \implicaLuego sumarApuestasIndividuo(\textit{apuestas}[i] = 1)) \y \\
        (\forall i,j: \ent)(0 \leq i <|{\textit{apuestas}}\vert \implicaLuego  0 \leq \textit{apuestas}[i][j] \geq 1) \y \\
        (\forall i,j: \ent)(0 \leq i <|{\textit{pagos}}\vert \y 0 \leq j <|{\textit{pagos}[i]}\vert \implicaLuego \textit{pagos}[i][j] > 0) \y \\
        (\forall i,j: \ent)(0 \leq i,j <|{\textit{eventos}}\vert \implicaLuego |{\textit{eventos}[i]}\vert = |{\textit{eventos}[j]}\vert) \y \\
        (\forall i,j: \ent)(0 \leq i < |{\textit{eventos}}\vert \y 0 \leq j < |{\textit{eventos}[i]}\vert \implicaLuego 0 \leq \textit{eventos}[i][j] \geq |{\textit{pagos}[i]}\vert)
        }
        \asegura{
            (\exists m:seq\langle seq\langle \real\rangle\rangle)(esApuestaVariante(m,\textit{individuo}) \yLuego (\exists \textit{trayectoriam}:seq\langle seq\langle \real\rangle\rangle) \\
            (esTrayectoria(\textit{trayectoriam},\textit{recursos},\textit{eventos},\textit{pagos},\textit{cooperan},m) \yLuego  \\
            (\forall A : seq\langle seq\langle \real\rangle\rangle)(esApuestaVariante(A, \textit{individuo}) \implicaLuego \\
            (\forall \textit{trayectoria} : seq\langle seq\langle \real\rangle\rangle)(esTrayectoria(\textit{trayectoria},\textit{recursos},\textit{eventos},\textit{pagos},\textit{cooperan},A) \implicaLuego \\
            \textit{trayectoriam}[\textit{individuo}][|{\textit{trayectoriam}[\textit{individuo}]}\vert-1] \geq \textit{trayectoria}[\textit{individuo}][|{\textit{trayectoria}[\textit{individuo}]}\vert-1]
            ))))
        }
    \end{proc}

    \pred{esApuestaVariante}{A : seq⟨seq⟨\real⟩⟩, \textit{individuo} : \nat}{
        |{A}\vert = |{\textit{$Apuestas_{0}$}}\vert \y (\forall i : \ent)(0 \leq i < |{\textit{$Apuestas_{0}$}}\vert \y i \neq \textit{individuo} \implicaLuego A[i] = \textit{$Apuestas_{0}$}[i]) \y \\
        |{A[\textit{individuo}]}\vert = |{\textit{$Apuestas_{0}$}[0]}\vert \y sumarApuestasIndividuo(A[\textit{individuo}]) = 1 \y \\
        (\forall j : \ent)(0 \leq j < |{A[\textit{individuo}]}\vert \implicaLuego 0 \leq A[\textit{individuo}][j] \leq 1)
    }

    \pred{esTrayectoria}{\textit{trayectoria} : seq⟨seq⟨\real⟩⟩,\textit{recursos} : seq⟨\real⟩,\textit{eventos} : seq⟨seq⟨\ent⟩⟩,\textit{pagos} : seq⟨seq⟨\real⟩⟩, \textit{cooperan} : seq⟨Bool⟩, \textit{apuestas} : seq⟨seq⟨\real⟩⟩}{
        primeroTieneARecursos(\textit{recursos}, \textit{trayectoria}) \hspace{2mm} \y \hspace{2mm} moduloEventosMasUno(\textit{eventos}, \textit{trayectoria}) \hspace{2mm} \y \\
        elementosDeT(\textit{apuestas}, \textit{pagos}, \textit{eventos}, \textit{cooperan}, \textit{trayectoria})
    }

\end{enumerate}

\clearpage

\section{Demostraciones de correctitud}
Demostrar que la siguiente especificación es correcta respecto de su implementación.
La función \textbf{frutoDelTrabajoPuramenteIndividual} calcula, para el ejemplo de apuestas al juego de cara o sello, cuánto
se ganaría si se juega completamente solo. Se contempla que el evento \textit{True} es cuando sale cara

\begin{proc}{frutoDelTrabajoPuramenteIndividual}{\In \textit{recursos}:seq⟨\real⟩, \In \textit{apuestas}:⟨s: \real, c:\real⟩,\In \textit{pago}:⟨s: \real, c:\real⟩,\In \textit{eventos}:seq⟨Bool⟩, \Out res: \real}{}

    \requiere { 
        \textit{$apuesta_{c}$ } + \textit{$apuesta_{s}$ } = 1 \y 
        \textit{$pago_{c}$}>0 \y 
        \textit{$pago_{s}$}>0 \y 
        \textit{$apuesta_{c}$} >0 \y 
        \textit{$apuesta_{s}$} >0 \y \textit{recurso} >0 
        }

    \asegura{\textit{res}=\textit{recurso}(\textit{$apuesta_{c}$}\textit{$pago_{c}$})^{\textit{\#apariciones}(\textit{eventos,T})(\textit{$apuesta_{s}$}\textit{$pago_{s}$})^{\textit{\#apariciones}(\textit{eventos,F})} }}

    \end{proc}

    Donde \textit{\#apariciones}(\textit{eventos, T}) es el auxiliar utilizado en la téorica, y \#(eventos, T) es su abreviación. \\
    
    res = \textit{recursos}

    i = 0
    
    While ($i < \left\lvert {\textit{eventos}}\right\rvert$) do
    
    \textcolor{white}{ea}if \textit{eventos}[i] then
    
    \textcolor{white}{easily}res = (res * $apuesta_{c}$ ) * $pago_{c}$
    
    \textcolor{white}{ea}else

    \textcolor{white}{easily}res = (res * $apuesta_{s}$ ) * $pago_{s}$

    \textcolor{white}{ea}endif

    \textcolor{white}{ea}i= i + 1

    endwhile \\
     
    Para probar la correctitud de este código usamos el teorema de correción de un ciclo, para lo cual se propone el siguiente invariante:

    $I \equiv 0 \leq i \leq |{\textit{eventos}}\vert \y \\
     \textit{res} = \textit{recurso}(\textit{$apuesta_{c}$}\textit{$pago_{c}$})^{(\sum_{j = 0}^{i -1})(if \hspace{0.5mm} eventos[j] = \True \hspace{0.5mm} then \hspace{0.5mm} 1 \hspace{0.5mm} else \hspace{0.5mm} 0 \hspace{0.5mm} fi)}
     (\textit{$apuesta_{s}$}\textit{$pago_{s}$})^{(\sum_{j = 0}^{i -1})(if \hspace{0.5mm} eventos[j] = \False \hspace{0.5mm} then  \hspace{0.5mm} 1  \hspace{0.5mm} else \hspace{0.5mm} 0 fi)}$\\
     
    Y la siguiente funcion decreciente:
    
    $\funcionDecresiente \equiv |\textit{eventos}\vert - i$ \\
    
    Entonces hay que probar:

    \begin{enumerate}
        \item $\precondicion \hspace{2mm} \implica \hspace{2mm} I$
        \item $\left\lbrace I \y B \right\rbrace \hspace{2mm} S \hspace{2mm} \left\lbrace I \right\rbrace$
        \item $I \y \neg B  \hspace{2mm} \implica \hspace{2mm} $ \postcondicion
        \item $\left\lbrace I \y B \y \metavariable = \funcionDecresiente \right\rbrace \hspace{2mm} S \hspace{2mm} \left\lbrace \funcionDecresiente < \metavariable \right\rbrace$
        \item $I \hspace{0.5mm} \y \hspace{0.5mm} \funcionDecresiente \leq 0  \hspace{2mm} \implica \hspace{2mm} \neg B$
    \end{enumerate} 
    
    
    \begin{enumerate}
        \item Veamos que se cumple 1 ($\precondicion \hspace{2mm} \implica \hspace{2mm} I$):\\


        $\precondicion \equiv \textit{res} = \textit{recursos} \y i=0$
        
        $I \equiv 0 \leq i \leq |{\textit{eventos}}\vert \y \\
        \textit{res} = \textit{recurso}(\textit{$apuesta_{c}$}\textit{$pago_{c}$})^{(\sum_{j = 0}^{i -1})(if \hspace{0.5mm} eventos[j] = \True \hspace{0.5mm} then \hspace{0.5mm} 1 \hspace{0.5mm} else \hspace{0.5mm} 0 \hspace{0.5mm} fi)}
        (\textit{$apuesta_{s}$}\textit{$pago_{s}$})^{(\sum_{j = 0}^{i -1})(if \hspace{0.5mm} eventos[j] = \False \hspace{0.5mm} then  \hspace{0.5mm} 1  \hspace{0.5mm} else \hspace{0.5mm} 0 fi)}$\\
        
        Podemos remplazar \precondicion \hspace{0.5mm} en I para ver que es tautologico. \\

        $I \equiv 0 \leq 0 \leq |{\textit{eventos}}\vert \y \\
        \textit{recurso} = \textit{recurso}(\textit{$apuesta_{c}$}\textit{$pago_{c}$})^{(\sum_{j = 0}^{0 -1})(if \hspace{0.5mm} eventos[j] = \True \hspace{0.5mm} then \hspace{0.5mm} 1 \hspace{0.5mm} else \hspace{0.5mm} 0 \hspace{0.5mm} fi)}
        (\textit{$apuesta_{s}$}\textit{$pago_{s}$})^{(\sum_{j = 0}^{0 -1})(if \hspace{0.5mm} eventos[j] = \False \hspace{0.5mm} then  \hspace{0.5mm} 1  \hspace{0.5mm} else \hspace{0.5mm} 0 fi)}$\\
        $\equiv \True \y \textit{res} = \textit{recurso}(\textit{$apuesta_{c}$}\textit{$pago_{c}$})^{0}
        (\textit{$apuesta_{s}$}\textit{$pago_{s}$})^{0}$\\   
        $\equiv \True \y \textit{recurso} = \textit{recurso} \equiv \True$
        \\

        Entonces se puede observar que $\precondicion \hspace{2mm} \implica \hspace{2mm} I$ \hspace{1mm} se cumple.

        \clearpage

        \item Veamos que se cumple 2 ($\left\lbrace I \y B \right\rbrace \hspace{2mm} S \hspace{2mm} \left\lbrace I \right\rbrace$):\\
        $I \equiv 0 \leq i \leq |{\textit{eventos}}\vert \y \\
        \textit{res} = \textit{recurso}(\textit{$apuesta_{c}$}\textit{$pago_{c}$})^{(\sum_{j = 0}^{i -1})(if \hspace{0.5mm} eventos[j] = \True \hspace{0.5mm} then \hspace{0.5mm} 1 \hspace{0.5mm} else \hspace{0.5mm} 0 \hspace{0.5mm} fi)}
        (\textit{$apuesta_{s}$}\textit{$pago_{s}$})^{(\sum_{j = 0}^{i -1})(if \hspace{0.5mm} eventos[j] = \False \hspace{0.5mm} then  \hspace{0.5mm} 1  \hspace{0.5mm} else \hspace{0.5mm} 0 fi)}$
        
        $ B: i < |{\textit{eventos}}\vert$
        
        Primero vemos si $I \y B \implica wp(if..., i := i + 1, I)$

        $wp(i := i +1, I) \equiv def(i+1) \yLuego \reemplazo \equiv (-1 \leq i \leq |{\textit{eventos}}\vert-1) \y \\
        (\textit{res} = \textit{recurso}(\textit{$apuesta_{c}$}\textit{$pago_{c}$})^{(\sum_{j = 0}^{i})(if \hspace{0.5mm} eventos[j] = \True \hspace{0.5mm} then \hspace{0.5mm} 1 \hspace{0.5mm} else \hspace{0.5mm} 0 \hspace{0.5mm} fi)}
        (\textit{$apuesta_{s}$}\textit{$pago_{s}$})^{(\sum_{j = 0}^{i})(if \hspace{0.5mm} eventos[j] = \False \hspace{0.5mm} then  \hspace{0.5mm} 1  \hspace{0.5mm} else \hspace{0.5mm} 0 fi)})\\
        \equiv E1$\\
        
        Entonces:\\
        $ wp(if..., E1) \equiv def(eventos[i]) \y (( eventos[i] \y wp(res := res\textit{\apuestac}\textit{\pagoc}, E1)) \lor \\ (\neg eventos[i] \y wp(res := res\textit{\apuestas}\textit{pagos}, E1)))$ 
        
        $\equiv (0 \leq i \leq |\textit{eventos}\vert) \y \\
        ((eventos[i] \y \\ res\apuestac\pagoc=\textit{recurso}(\textit{\apuestac}\textit{\pagoc})^{(\sum_{j = 0}^{i})(if eventos[j] = \True  then 1 else 0  fi)}(\textit{\apuestas}\textit{\pagos})^{(\sum_{j = 0}^{i})(if eventos[j] = \False then  1 else 0 fi)}) \\
        \lor \\ (\neg eventos[i] \y \\ res\apuestas\pagos=\textit{recurso}(\textit{\apuestac}\textit{\pagoc})^{(\sum_{j = 0}^{i})(if eventos[j] = \True  then 1 else 0  fi)}(\textit{\apuestas}\textit{\pagos})^{(\sum_{j = 0}^{i})(if eventos[j] = \False then 1 else 0 fi)})\\
        ) $
        
        Luego, en la ecuacion:\\
        $res\apuestac\pagoc=\textit{recurso}(\textit{\apuestac}\textit{\pagoc})^{(\sum_{j = 0}^{i})(if eventos[j] = \True  then 1 else 0  fi)}(\textit{\apuestas}\textit{\pagos})^{(\sum_{j = 0}^{i})(if eventos[j] = \False then  1 else 0 fi)}$\\
        se puede pasar dividiendo el término $\apuestac\pagoc$ dividiendo, ya que tanto \textit{$apuesta_{c}$} como \textit{$pago_{c}$} son mayores a 0 por lo pedido en el requiere.\\
        Lo mismo ocurre en la ecuacion:\\
        $res\apuestas\pagos=\textit{recurso}(\textit{\apuestac}\textit{\pagoc})^{(\sum_{j = 0}^{i})(if eventos[j] = \True  then 1 else 0  fi)}(\textit{\apuestas}\textit{\pagos})^{(\sum_{j = 0}^{i})(if eventos[j] = \False then 1 else 0 fi)}$
        donde se puede pasar dividiendo el término $\apuestas\pagos$ dividiendo por la misma razón.\\

        De esta forma, se obtiene la siguiente expresión:\\
        $wp(if..., E1) \equiv (0 \leq i \leq |\textit{eventos}\vert) \y \\
        ((eventos[i] \y \\ res=\textit{recurso}(\textit{$apuesta_{c}$}\textit{$pago_{c}$})^{(\sum_{j = 0}^{i-1})(if eventos[j] = \True  then 1 else 0  fi)}(\textit{$apuesta_{s}$}\textit{$pago_{s}$})^{(\sum_{j = 0}^{i})(if eventos[j] = \False then  1 else 0 fi)}) 
        \lor \\ (\neg eventos[i] \y \\ res=\textit{recurso}(\textit{$apuesta_{c}$}\textit{$pago_{c}$})^{(\sum_{j = 0}^{i})(if eventos[j] = \True  then 1 else 0  fi)}(\textit{$apuesta_{s}$}\textit{$pago_{s}$})^{(\sum_{j = 0}^{i-1})(if eventos[j] = \False then 1 else 0 fi)})
        ) 
        $\\

        Por otro lado, tambien cabe destacar que en las ecuaciones:\\
        $res=\textit{recurso}(\textit{$apuesta_{c}$}\textit{$pago_{c}$})^{(\sum_{j = 0}^{i-1})(if eventos[j] = \True  then 1 else 0  fi)}(\textit{$apuesta_{s}$}\textit{$pago_{s}$})^{(\sum_{j = 0}^{i})(if eventos[j] = \False then  1 else 0 fi)}$ 
        y\\
        $res=\textit{recurso}(\textit{$apuesta_{c}$}\textit{$pago_{c}$})^{(\sum_{j = 0}^{i})(if eventos[j] = \True  then 1 else 0  fi)}(\textit{$apuesta_{s}$}\textit{$pago_{s}$})^{(\sum_{j = 0}^{i-1})(if eventos[j] = \False then 1 else 0 fi)}$\\
        se pueden reemplazar\\ 
        $(\textit{$apuesta_{s}$}\textit{$pago_{s}$})^{(\sum_{j = 0}^{i})(if eventos[j] = \False then  1 else 0 fi)}$ por 
        $(\textit{$apuesta_{s}$}\textit{$pago_{s}$})^{(\sum_{j = 0}^{i-1})(if eventos[j] = \False then  1 else 0 fi)}$ 
        y\\
        $(\textit{$apuesta_{c}$}\textit{$pago_{c}$})^{(\sum_{j = 0}^{i})(if eventos[j] = \True  then 1 else 0  fi)}$ por
        $(\textit{$apuesta_{c}$}\textit{$pago_{c}$})^{(\sum_{j = 0}^{i-1})(if eventos[j] = \True  then 1 else 0  fi)}$\\
        respectivamente, ya que en que ambos casos, el último elemento es igual a 0 y por ende no afecta al resultado de la sumatoria. \\
        
        De ahí que se obtiene la siguiente expresión:\\
        $wp(if..., E1) \equiv (0 \leq i \leq |\textit{eventos}\vert) \y \\
        ((eventos[i] \y \\ res=\textit{recurso}(\textit{$apuesta_{c}$}\textit{$pago_{c}$})^{(\sum_{j = 0}^{i-1})(if eventos[j] = \True  then 1 else 0  fi)}(\textit{$apuesta_{s}$}\textit{$pago_{s}$})^{(\sum_{j = 0}^{i-1})(if eventos[j] = \False then  1 else 0 fi)}) 
        \lor \\ (\neg eventos[i] \y \\ res=\textit{recurso}(\textit{$apuesta_{c}$}\textit{$pago_{c}$})^{(\sum_{j = 0}^{i-1})(if eventos[j] = \True  then 1 else 0  fi)}(\textit{$apuesta_{s}$}\textit{$pago_{s}$})^{(\sum_{j = 0}^{i-1})(if eventos[j] = \False then 1 else 0 fi)})
        ) 
        $

        $\equiv (0 \leq i \leq |\textit{eventos}\vert) \y 
        ((eventos[i] \lor \neg eventos[i]) \y \\
        (res = \textit{recurso}(\textit{$apuesta_{c}$}\textit{$pago_{c}$})^{(\sum_{j = 0}^{i-1})(if eventos[j] = \True  then 1 else 0  fi)}(\textit{$apuesta_{s}$}\textit{$pago_{s}$})^{(\sum_{j = 0}^{i-1})(if eventos[j] = \False then 1 else 0 fi)})
        ) 
        $

        $\equiv (0 \leq i \leq |\textit{eventos}\vert) \y \\
        (res = \textit{recurso}(\textit{$apuesta_{c}$}\textit{$pago_{c}$})^{(\sum_{j = 0}^{i-1})(if eventos[j] = \True  then 1 else 0  fi)}(\textit{$apuesta_{s}$}\textit{$pago_{s}$})^{(\sum_{j = 0}^{i-1})(if eventos[j] = \False then 1 else 0 fi)})
        $

        $\equiv I $

        Entonces se puede observar que $\left\lbrace I \y B \right\rbrace \hspace{2mm} S \hspace{2mm} \left\lbrace I \right\rbrace$ \hspace{1mm} se cumple.

        \clearpage

        \item Veamos que se cumple 3 ($I \y \neg B \hspace{2mm} \implica \hspace{2mm} $ \postcondicion):\\
        
        $\postcondicion: \textit{res}=\textit{recurso}(\textit{$apuesta_{c}$}\textit{$pago_{c}$})^{\textit{\#apariciones}(\textit{eventos,T})}(\textit{$apuesta_{s}$}\textit{$pago_{s}$})^{\textit{\#apariciones}(\textit{eventos,F})}$

        $\neg B: i \geq |{\textit{eventos}}\vert$

        $I \y \neg B \hspace{2mm} \implica \hspace{2mm} $ \postcondicion

        $I \y \neg B \equiv  0 \leq i \leq |\textit{eventos}\vert \y i \geq |\textit{eventos}\vert \\
        \textit{res} = \textit{recurso}(\textit{$apuesta_{c}$}\textit{$pago_{c}$})^{(\sum_{j = 0}^{i -1})(if \hspace{0.5mm} eventos[j] = \True \hspace{0.5mm} then \hspace{0.5mm} 1 \hspace{0.5mm} else \hspace{0.5mm} 0 \hspace{0.5mm} fi)}
        (\textit{$apuesta_{s}$}\textit{$pago_{s}$})^{(\sum_{j = 0}^{i -1})(if \hspace{0.5mm} eventos[j] = \False \hspace{0.5mm} then  \hspace{0.5mm} 1  \hspace{0.5mm} else \hspace{0.5mm} 0 fi)}$\\

        $I \y \neg B \equiv  i = |\textit{eventos}\vert \y \\
        \textit{res} = \textit{recurso}(\textit{$apuesta_{c}$}\textit{$pago_{c}$})^{(\sum_{j = 0}^{|\textit{eventos}\vert -1})(if \hspace{0.5mm} eventos[j] = \True \hspace{0.5mm} then \hspace{0.5mm} 1 \hspace{0.5mm} else \hspace{0.5mm} 0 \hspace{0.5mm} fi)}
        (\textit{$apuesta_{s}$}\textit{$pago_{s}$})^{(\sum_{j = 0}^{|\textit{eventos}\vert -1})(if \hspace{0.5mm} eventos[j] = \False \hspace{0.5mm} then  \hspace{0.5mm} 1  \hspace{0.5mm} else \hspace{0.5mm} 0 fi)}$

        $I \y \neg B \equiv  i = |\textit{eventos}\vert \y 
        \textit{res}=\textit{recurso}(\textit{$apuesta_{c}$}\textit{$pago_{c}$})^{\textit{\#apariciones}(\textit{eventos,T})}(\textit{$apuesta_{s}$}\textit{$pago_{s}$})^{\textit{\#apariciones}(\textit{eventos,F})}$\\
        
        Entonces se puede observar que $I \y \neg B \hspace{2mm} \implica \hspace{2mm} \postcondicion$  \hspace{1mm} se cumple.

        

        \item Veamos que se cumple 4 ($\left\lbrace I \y B \y \metavariable = \funcionDecresiente \right\rbrace \hspace{2mm} S \hspace{2mm} \left\lbrace \funcionDecresiente < \metavariable \right\rbrace$): \\
        
        $I \equiv 0 \leq i \leq |{\textit{eventos}}\vert \y \\
        \textit{res} = \textit{recurso}(\textit{$apuesta_{c}$}\textit{$pago_{c}$})^{(\sum_{j = 0}^{i -1})(if \hspace{0.5mm} \textit{eventos}[j] = \True \hspace{0.5mm} then \hspace{0.5mm} 1 \hspace{0.5mm} else \hspace{0.5mm} 0 \hspace{0.5mm} fi)}
        (\textit{$apuesta_{s}$}\textit{$pago_{s}$})^{(\sum_{j = 0}^{i -1})(if \hspace{0.5mm} \textit{eventos}[j] = \False \hspace{0.5mm} then  \hspace{0.5mm} 1  \hspace{0.5mm} else \hspace{0.5mm} 0 fi)}$\\
        $B \equiv i < |\textit{eventos}\vert$\\
       \funcionDecresiente= $|\textit{evento}\vert - i$\\
       $I \y B \y \metavariable = \funcionDecresiente\\
       \equiv 0 \leq i \leq |{\textit{eventos}}\vert \y \\
       \textit{res} = \textit{recurso}(\textit{$apuesta_{c}$}\textit{$pago_{c}$})^{(\sum_{j = 0}^{i -1})(if \hspace{0.5mm} \textit{eventos}[j] = \True \hspace{0.5mm} then \hspace{0.5mm} 1 \hspace{0.5mm} else \hspace{0.5mm} 0 \hspace{0.5mm} fi)}
       (\textit{$apuesta_{s}$}\textit{$pago_{s}$})^{(\sum_{j = 0}^{i -1})(if \hspace{0.5mm} \textit{eventos}[j] = \False \hspace{0.5mm} then  \hspace{0.5mm} 1  \hspace{0.5mm} else \hspace{0.5mm} 0 fi)}\y \\
       i < |\textit{eventos}\vert \y |\textit{evento}\vert - i = \metavariable$ \\
       $\equiv 0 \leq i < |{\textit{eventos}}\vert \y |\textit{evento}\vert - i = \metavariable \y \\
       \textit{res} = \textit{recurso}(\textit{$apuesta_{c}$}\textit{$pago_{c}$})^{(\sum_{j = 0}^{i -1})(if \hspace{0.5mm} \textit{eventos}[j] = \True \hspace{0.5mm} then \hspace{0.5mm} 1 \hspace{0.5mm} else \hspace{0.5mm} 0 \hspace{0.5mm} fi)}
       (\textit{$apuesta_{s}$}\textit{$pago_{s}$})^{(\sum_{j = 0}^{i -1})(if \hspace{0.5mm} \textit{eventos}[j] = \False \hspace{0.5mm} then  \hspace{0.5mm} 1  \hspace{0.5mm} else \hspace{0.5mm} 0 fi)}$\\

       Primero hay que buscar la precondición más débil del ciclo: \\
       $wp(i := i + 1,{\funcionDecresiente < \metavariable}) \equiv def(i+1) \y Q_{i+1}^{i} \equiv def(i) \y |\textit{evento}\vert - (i+1) < \metavariable
       \equiv  def(i) \y |\textit{evento}\vert - i \leq \metavariable$ \\

       Luego tenemos la siguiente expresión:\\
       $ wp((if \hspace{1mm}  \textit{eventos}[i] \hspace{1mm} then  \hspace{1mm} res=(res*\textit{$apuesta_{c}$})\textit{$pago_{c}$} \hspace{1mm} else \hspace{1mm} res=(res*\textit{$apuesta_{s}$})\textit{$pago_{s}$} \hspace{1mm} fi), Q)
       $\\
       $\equiv def(\textit{eventos}[i]) \yLuego 
       ((\textit{eventos}[i] \y wp(\textit{res} := \textit{res}(\textit{$apuesta_{c}$}\textit{$pago_{c}$}), Q)) \lor 
       (\neg\textit{eventos}[i] \y wp(\textit{res} := \textit{res}(\textit{$apuesta_{s}$}\textit{$pago_{s}$}), Q))) 
       $\\
       $\equiv 0 \leq i < |\textit{evento}\vert \yLuego 
       ((\textit{eventos}[i] \y def(\textit{res}(\textit{$apuesta_{c}$}\textit{$pago_{c}$})) \y (def(i) \y |\textit{evento}\vert - i \leq \metavariable)) \lor \\
       (\neg\textit{eventos}[i] \y def(\textit{res}(\textit{$apuesta_{s}$}\textit{$pago_{s}$})) \y (def(i) \y |\textit{evento}\vert - i \leq \metavariable)))
       $\\
       $\equiv 0 \leq i < |\textit{evento}\vert \yLuego 
       ((\textit{eventos}[i] \y def(\textit{res}) \y def(\textit{$apuesta_{c}$}) \y def(\textit{$pago_{c}$}) \y def(i) \y |\textit{evento}\vert - i \leq \metavariable) \lor \\
       (\neg\textit{eventos}[i] \y def(\textit{res}) \y def(\textit{$apuesta_{s}$}) \y def(\textit{$pago_{s}$}) \y def(i) \y |\textit{evento}\vert - i \leq \metavariable))
       $\\
       $\equiv 0 \leq i < |\textit{evento}\vert \yLuego 
       ((\textit{eventos}[i] \y |\textit{evento}\vert - i \leq \metavariable) \lor
       (\neg\textit{eventos}[i] \y |\textit{evento}\vert - i \leq \metavariable))
       $\\
       $\equiv  0 \leq i < |\textit{evento}\vert \yLuego ((\textit{eventos}[i] \lor \neg\textit{eventos}[i]) \y |\textit{evento}\vert - i \leq \metavariable)
       $\\
       $\equiv 0 \leq i < |\textit{evento}\vert \y |\textit{evento}\vert - i \leq \metavariable
       $\\

       Finalmente, se puede observar: \\
       $(0 \leq i < |{\textit{eventos}}\vert \y |\textit{evento}\vert - i = \metavariable \y \\
       \textit{res} = \textit{recurso}(\textit{$apuesta_{c}$}\textit{$pago_{c}$})^{(\sum_{j = 0}^{i -1})(if \hspace{0.5mm} \textit{eventos}[j] = \True \hspace{0.5mm} then \hspace{0.5mm} 1 \hspace{0.5mm} else \hspace{0.5mm} 0 \hspace{0.5mm} fi)}
       (\textit{$apuesta_{s}$}\textit{$pago_{s}$})^{(\sum_{j = 0}^{i -1})(if \hspace{0.5mm} \textit{eventos}[j] = \False \hspace{0.5mm} then  \hspace{0.5mm} 1  \hspace{0.5mm} else \hspace{0.5mm} 0 fi)}) \\
       \implica (0 \leq i < |{\textit{eventos}}\vert \y |\textit{evento}\vert - i = \metavariable) \implica (0 \leq i < |\textit{evento}\vert \y |\textit{evento}\vert - i \leq \metavariable)
       $\\
       y por ende, $\left\lbrace I \y B \y \metavariable = \funcionDecresiente \right\rbrace \hspace{2mm} S \hspace{2mm} \left\lbrace \funcionDecresiente < \metavariable \right\rbrace$ se cumple.\\

        \item Veamos que se cumple 5 ($I \hspace{0.5mm} \y \hspace{0.5mm} \funcionDecresiente \leq 0  \hspace{2mm} \implica \hspace{2mm} \neg B$):\\
        
        $\funcionDecresiente \equiv |\textit{eventos}\vert - i$

        $\neg B \equiv i \geq |{\textit{eventos}}\vert$

        $I \hspace{0.5mm} \y \hspace{0.5mm} \funcionDecresiente \leq 0  \hspace{2mm} \equiv I \y  |\textit{eventos}\vert - i \leq 0 $
        $\equiv I \y |\textit{eventos}\vert \leq i \equiv I \y \neg B$

        Luego, $I \y \neg B \implica \neg B$

        Entonces se puede observar que $I \hspace{0.5mm} \y \hspace{0.5mm} \funcionDecresiente \leq 0  \hspace{2mm} \implica \hspace{2mm} \neg B$ se cumple.
        

    \end{enumerate}
\end{document}